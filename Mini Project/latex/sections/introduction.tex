

\section{Introduction}

\footnotetext[1]{Project code: \href{https://github.com/kanon-saint/CMSC-197/blob/7c74a7ac39b88238cca9b42eebdca0725433e354/Mini\%20Project/Sentiment\%20Analysis\%20of\%20Emotions.ipynb}{\color{blue}GitHub Repository Link}}

\footnotetext[2]{Dataset: \href{https://www.kaggle.com/datasets/praveengovi/emotions-dataset-for-nlp/data}{\color{blue}Kaggle Dataset Link}}

Social media platforms like Twitter are among the most popular social networks worldwide, with a significant number of monthly active users as of April 2024. It ranks 12th globally by monthly active users \cite{statistica2023}. Twitter has emerged as an influential platform for individuals to express their thoughts, opinions, and emotions. With over 500 million tweets posted daily and 237.8 million daily active users, Twitter offers a unique opportunity to study human emotions on a large scale \cite{internetlivestats2023}\cite{twitterstats2023}. As public expression increasingly occurs in social media platforms, understanding the emotional content of social media posts becomes crucial for various fields, including sentiment analysis (SA), mental health, and public opinion research.

Emotions are very important in human communication, and social media platforms provide a rich source of emotional data. Emotion analysis has become an important area of research in the broader field of SA, where the emotion of textual data is analyzed to understand human behavior \cite{Kim2019}. It is significant in understanding the affective dimension of literature, but its applications extend to various domains, including social media, where user-generated content offers insights into public sentiment, mental well-being, and social interactions \cite{Hakak2017}. The growth of social media platforms like Twitter has further fueled the interest in this research, with vast amounts of text-based data available for analysis, enabling the study of human emotions on large scale.

As technology advances, the ability of computers to recognize and respond to human emotions has become increasingly important. Emotion detection, often tied closely to SA, is a key component of human-computer interaction (HCI). It enables systems to adjust their behavior based on users' emotional states, improving user experience and satisfaction. Machine learning (ML) techniques, particularly supervised learning approaches such as Support Vector Machines (SVM) and Naive Bayes, have proven effective in detecting emotions across various data sources, including text, speech, and even biosignals. These techniques have been applied successfully in different domains, ranging from mental health to interactive systems design. However, challenges remain in standardizing datasets and evaluation procedures, as well as in expanding the scope beyond English-language data sources \cite{Alslaity2022}.
